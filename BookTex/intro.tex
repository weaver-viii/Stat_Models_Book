

\chapter*{Introduction: Why study applied statistics}
\label{ch:intro}



Advances in information technology have made the process of collecting huge amounts of data not only possible, but also extremely easy. The massive quantities of data available in the modern era enable scientists, social scientists, government agencies and companies to ask and answer questions in order to understand the physical world, make pubic policies, and improve productivity. For example, data can now be used to provide answers to an extremely broad array of questions such as ``is there a link between alcohol and birth defects?'' and ``What effect does incarceration have on the probability of a re-offence?''. Statistics is an indispensable tool in the process of extracting meaningful answers to the questions being asked based on the collected data. In fact, statistics is more than an analysis tool: statistical experimental design provides principles and methods for ensuring that the data is collected in such a way that facilitates the investigators' ability to effectively address the questions asked.

The tools provided by statistics were first developed as instruments used to solve real-world problems such as demographic census collection and agricultural experiments. As the types of data being collected grew more intricate, detailed and complex, so too did the accompanying statistical methods and the assumptions upon which they are based. Consequently, analyzing a dataset with no real-world knowledge of how the data was collected or what the numbers represent is the equivalent to \textcolor{red}{some nice analogy}. 


To ensure that useful questions are asked and valid answers are drawn, domain experts (scientists, for example) should always be in the picture for statisticians to strengthen their domain knowledge by asking questions and brainstorming. Data does not stand alone: it must be viewed with a strong consideration towards the questions being asked, the scientific background of the problem and the data collection process.



\section*{What to expect from this book}

The primary purpose of this book is to demonstrate what it is like to use statistics in the real world and to how to work together with people who have domain expertise in order to do data analysis with the aim of answering questions outside statistics. As a reader of this book, you will gain insight into the many steps involved in the iterative process of extracting information from data for the purposes of prediction and interpretation. In particular, this book will provide the necessary background to cover basic useful statistical methods in practice and will highlight how judgement and common sense should be used in the process of data analysis and when arriving at valid conclusions. 

Statistical techniques will be introduced through a first-principles approach with a focus on developing intuition, and the reader will learn how to develop appropriate techniques in unfamiliar situations. A number of data analysis labs are provided which are designed to help the reader put into practice the techniques and intuition conveyed throughout the book.

These principles will be demonstrated in the context of an ongoing collaborative genomic \emph{Drosophila} fruitfly project I am working on with several of my students in statistics and my colleagues in the biological sciences. 

\section*{Who should read this book}

The reader of this book is assumed to be familiar with concepts typically introduced in upper division mathematical statistics and probability courses. Further, to get the most out of the labs, the reader is encouraged to have familiarity with writing functions, manipulating and cleaning data and creating graphics in a programming language such as R.

This book is ideal for those who are looking to learn how to connect statistics to the real world, not those who are seeking the "optimal" method for broad classes of problems. If you are familiar with statistical concepts but would like to further develop your ability to use common sense, judgement and critical thinking in statistics, then this book is for you.


\section*{Recommended reading}

There exist a number of excellent texts for learning the intuition behind and the application of statistics. We will touch on a few here, namely Statistical Models: Theory and Practice by David A. Freedman, the sentiments of which we have tried to echo in this book. Another great read (not just for those new to statistics, but also for the most seasoned analysts) is Statistics by David A. Freedman, Robert Pisani and Roger Purves. This book does a fantastic job of teaching statistics through connections to reality while managing to completely stay away from the mathematics all together.